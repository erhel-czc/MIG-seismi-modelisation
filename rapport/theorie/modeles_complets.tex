\documentclass[10pt]{article}
\usepackage{graphicx} % Required for inserting images
\usepackage{geometry}
\usepackage{amsmath}
\geometry{hmargin=2.5cm,vmargin=1.5cm}

\title{Modélisation d'une faille à proximité d'une zone d'injection de fluide (géothermie profonde)}
\author{Léonard Duval-Laude}
\date{Novembre 2025}

\begin{document}

\maketitle

\section{Modèle avec la pression de fluide (cas horizontal $\phi = \pi/2$)}

\begin{figure}[h]
    \centering
    \includegraphics[width=0.5\linewidth]{schema_final.png}
    \caption{Modélisation d'une faille située à une distance r du point où le fluide est injecté par la centrale de géothermie profonde.}
    \label{fig:placeholder}
\end{figure}

D'après l'article "JGR Solid Earth - 2015 - Segall - Injection induced seismicity  Poroelastic and earthquake nucleation effects", on a l'expression de la pression fluide en fonction du temps t à une dsitance r du point d'injection (où $q$ est le débit volumique, $K$ la perméabilité du milieu, $\eta$ la viscosité dynamique et $\rho}_0$ la masse volumique du fluide) :

\begin{equation}
P(t)
= \frac{q }{4\pi{\rho}_0 r}\frac{\eta}{K}\operatorname{erfc}(\frac{r}{2\sqrt{ct}})=P_{\infty}\operatorname{erfc}(\sqrt{\frac{\tau}{t}})
\end{equation}

avec : $$P_{\infty}=\frac{q }{4\pi{\rho}_0 r}\frac{\eta}{K}, \qquad
\tau= \frac{r^2}{4c}$$

En l'absence de pression fluide, seule la pression lithostatique $\sigma_0$ s'exerce:
$\sigma_0 = \rho g h$ ($\rho$ la masse volumique moyenne de la croûte continentale)
\\

La pression de fluide "soulage" la contrainte normale ($\sigma_{eff}=\sigma_0-P(t)$) et on a:
\begin{equation}
\eta v 
= -k\left( l^{*} + \delta - v_p t \right)
  - f(t)(\sigma_0 -P(t))
\end{equation}

On dérive par rapport au temps :
\begin{equation}
\eta \frac{dv}{dt}
= -k (v - v_p)
  - \left(\frac{df}{dt}(\sigma_0 -P(t))-\frac{dP}{dt}f(t)\right)
\end{equation}

On multiplie par $\frac{d_c}{v_p b \sigma_0}$ (idem que le cas sans pression de fluide) :

\begin{equation}
\left(\bar{\eta}+\frac{\alpha}{\bar{v}}(1-\frac{P(t)}{\sigma_0})\right) \frac{d\bar{v}}{d\bar{t}}
=
-\kappa (\bar{v} - 1) + (\bar{v}-\frac{1}{\bar{\theta}})(1-\frac{P(t)}{\sigma_0})
  + \left(f_0+a\ln{\bar{(v)}}+b\ln{\bar{(\theta)}}\right)\frac{d_c}{v_p b \sigma_0}\frac{dP}{dt}
\end{equation}



Or, on a :

\begin{equation}
\frac{P(t)}{\sigma_0}=\bar{P_{\infty}}\operatorname{erfc}(\frac{\bar{r}}{2\sqrt{\bar{c}\bar{t}}})=\bar{P_{\infty}}\operatorname{erfc}(\frac{\bar{\xi}}{2})
\end{equation}

\begin{equation} 
\frac{d_c}{v_p b \sigma_0}\frac{dP}{dt} =\frac{1}{b}\frac{1}{2\sqrt{\pi}}\frac{\bar{P_{\infty}}}{\bar{t}}\frac{\bar{r}}{\sqrt{\bar{c}\bar{t}}}
e^{-\left(\frac{\bar{r}}{2\sqrt{\bar{c}\bar{t}}}\right)^2}=\frac{1}{b}\frac{1}{2\sqrt{\pi}}\frac{\bar{P_{\infty}}}{\bar{t}}\bar{\xi}
e^{-(\frac{\bar{\xi}}{2})^2}
\end{equation}

Les variables adimensionnées étant:

$$
\bar{r} = \frac{r}{L^{\star}}=\frac{r}{\frac{\mu d_c}{b \sigma_0}}, \qquad 
\bar{c} = c\frac{b^2\sigma_0^2}{v_p \mu^2 d_c}, \qquad 
\bar{P_{\infty}}=\frac{P_{\infty}}{\sigma_0},\qquad 
\bar{\xi}=\frac{\bar{r}}{\sqrt{\bar{c}\bar{t}}}$
$$

On en déduit la formule utilisée dans le programme Python :

\begin{equation}
\fbox{\[
\displaystyle \frac{-\kappa(\bar{v} - 1)
  +(\bar{v}
  - \frac{1}{\bar{\theta}})(1-\bar{P_{\infty}}\operatorname{erfc}(\frac{\bar{\xi}}{2})+\left(\frac{f_0}{b}+\alpha\ln{\bar{(v)}}+\ln{\bar{(\theta)}}\right)\frac{\bar{P_{\infty}}}{\bar{t}}\frac{\bar{\xi}}{2\sqrt{\pi}}e^{-(\frac{\bar{\xi}}{2})^2}}{\bar{\eta}\bar{v}+\alpha(1-\bar{P_{\infty}}\operatorname{erfc}(\frac{\bar{\xi}}{2}))}d\bar{t} = \frac{d\bar{v}}{\bar{v}} \]} 
\end{equation}

\section{Modèle avec la pression de fluide (angle $\phi$ quelconque)}

L'équation (3) devient alors:

\begin{equation}
\eta \frac{dv}{dt}
= -k (v \sin{(\phi)} - v_p)\sin{(\phi)}
  - \left(\frac{df}{dt}(\sigma(t) -P(t))+\frac{d(\sigma-P)}{dt}f(t)\right)
\end{equation}

L'équation finale précédente (7) devient:

\begin{equation}
\fbox{\[
\displaystyle \frac{-\kappa(\bar{v} \sin{(\phi)} - 1)\sin{(\phi)}
  +(\bar{v}
  - \frac{1}{\bar{\theta}})(\frac{\sigma(t)}{\sigma_0}-\frac{P(t)}{\sigma_0})-\left(\frac{f_0}{b}+\alpha\ln{\bar{(v)}}+\ln{\bar{(\theta)}}\right)\frac{d(\sigma -P)}{dt}\frac{d_c}{v_p \sigma_0}}{\bar{\eta}\bar{v}+\alpha(\frac{\sigma(t)}{\sigma_0}-\frac{P(t)}{\sigma_0})}d\bar{t} = \frac{d\bar{v}}{\bar{v}} \]} 
\end{equation}

Sachant que :
\begin{equation}
\sigma(t)= \sigma_0\sin{(\phi)} -k(v_p t-\sin{(\phi)}\delta-l^{\star})\cos{\phi}
\end{equation}

\section{Prise en compte de la poro-élasticité du milieu}

\begin{figure}[h]
    \centering
    \includegraphics[width=0.5\linewidth]{schema_2_bon.png}
    \caption{Modélisation d'une faille située à une distance r du point où le fluide est injecté par la centrale de géothermie profonde.}
    \label{fig:placeholder}
\end{figure}

D'après l'article "JGR Solid Earth - 2015 - Segall - Injection induced seismicity  Poroelastic and earthquake nucleation effects", la prise e compte de la poro-élasticité du milieu se traduit par l'ajout d'un tenseur des contraintes $\underline{\sigma}$, dont on peut en déduire la traction $\vec{T}$ associé:


\begin{flalign*}
&\vec{T} =\underline{\sigma} \vec{e_y} \\
&\vec{T}= \left( (\sigma_{XX}-\sigma_{YY})\frac{\sin{(2\phi)}}{2}-\sigma_{XY}\cos{(2\phi)} \right)\vec{e_x} + \left( \sigma_{XY}\sin{(2\phi)}+\sigma_{XX}\cos{(\phi)}^2+\sigma_{YY}\sin{(\phi)^2} \right)\vec{e_y}\\
&\vec{T} = \sigma_{pe}^{(x)} \vec{e_x} + \sigma_{pe}^{(y)} \vec{e_y}\\
\end{flalign*}

\begin{equation}
\sigma_{ij}(t)
= -\, \frac{ q(\lambda_u - \lambda)\,\mu }
       { 4\pi \rho_0 c\, r\, \alpha\, (\lambda_u + 2\mu) }
\left\{
\delta_{ij}\!\left[
\operatorname{erfc}\!\!\left(\frac{\xi}{2}\right)
- 2\frac{g(\xi)}{\xi^2}
\right]
+
\frac{X_i X_j}{r^2}
\left[
\operatorname{erfc}\!\!\left(\frac{\xi}{2}\right)
+ 6\frac{g(\xi)}{\xi^2}
\right]
\right\}.
\end{equation}
L'équation (8) devient alors:

\begin{equation}
\eta \frac{dv}{dt}
= -k (v \sin{(\phi)} - v_p)\sin{(\phi)}
  - \left(\frac{df}{dt}\left(\sigma(t)+\sigma_{pe}^{(y)}(t) -P(t)\right)+\frac{d(\sigma+\sigma_{pe}^{(y)}-P)}{dt}f(t)\right)+\frac{d\sigma_{pe}^{(x)}}{dt}
\end{equation}

\end{document}