\documentclass{article}
\usepackage{graphicx} % Required for inserting images

\title{Adimensionnement}
\author{Léonard Duval-laude}
\date{Novembre 2025}

\begin{document}

\maketitle

\section{Introduction}

% Définitions
$$
\eta = \frac{\mu}{2\beta}, \qquad 
\beta = \sqrt{\frac{\mu}{\rho}}
$$
% Équations
\begin{equation}
\eta v 
= -k\left( l^{*} + \delta - v_p t \right)
  - f_0 \sigma
  - a \sigma \ln\!\left( \frac{v(t)}{v_p} \right)
  - b \sigma \ln\!\left( \frac{v_p\, \theta(t)}{d_c} \right)
\end{equation}

On dérive par rapport au temps :
\begin{equation}
\eta \frac{dv}{dt}
= -k (v - v_p)
  - a \sigma \frac{1}{v} \frac{dv}{dt}
  - b \sigma \frac{1}{\theta} \frac{d\theta}{dt}
\end{equation}

On pose les variables adimensionnées suivantes et on obtient:
$$
\bar{t} = \frac{t v_p}{d_c}, \qquad 
\bar{v}=\frac{v}{v_p}, \qquad 
\bar{\theta}=\frac{\theta v_p}{d_c}
$$

\begin{equation}
\eta \frac{dv}{dt}
= -k v_p(\bar{v} - 1)
  - \frac{v_p a \sigma}{d_c}\frac{1}{\bar{v}} \frac{d\bar{v}}{d\bar{t}}
  - \frac{v_p b \sigma}{d_c} \frac{1}{\bar{\theta}} \frac{d\bar{\theta}}{d\bar{t}}
\end{equation}

On multiplie par $\frac{d_c}{v_p b \sigma}$ :

\begin{equation}
\frac{\eta v_p}{b \sigma} \frac{d_c}{{v_p}^2} \frac{dv}{dt}
= -\kappa(\bar{v} - 1)
  - \alpha \frac{1}{\bar{v}} \frac{d\bar{v}}{d\bar{t}}
  - \frac{1}{\bar{\theta}} \frac{d\bar{\theta}}{d\bar{t}}
\end{equation}

\begin{center}
\[ \left\{
\begin{array}{l}
  \kappa = \frac{k d_c}{b \sigma} \\
  \alpha = \frac{a}{b}
\end{array}
\right.
\]
\end{center}

Comme $\frac{d \bar{v}}{d \bar{t}}=\frac{d_c}{{v_p}^2}\frac{dv}{dt}$ et en posant $\bar{\eta}=\frac{\eta v_p}{b \sigma}$, on obtient:

\begin{equation}
\bar{\eta}\frac{d \bar{v}}{d \bar{t}} 
= -\kappa(\bar{v} - 1)
  - \alpha \frac{1}{\bar{v}} \frac{d\bar{v}}{d\bar{t}}
  - \frac{1}{\bar{\theta}} \frac{d\bar{\theta}}{d\bar{t}}
\end{equation}

Or $\frac{d\bar{\theta}}{d\bar{t}}=\frac{d\theta}{dt}=1-\frac{v\theta}{d_c}=1-\bar{v} \bar{\theta}$ on a finalement :

\begin{equation}
\frac{d \bar{v}}{d \bar{t}}(\bar{\eta}+\frac{\alpha}{\bar{v}})
= -\kappa(\bar{v} - 1)
  +\bar{v}
  - \frac{1}{\bar{\theta}}
\end{equation}
On en déduit la formule utilisée dans le programme Python :

\begin{equation}
\fbox{\[
\displaystyle \frac{-\kappa(\bar{v} - 1)
  +\bar{v}
  - \frac{1}{\bar{\theta}}}{\alpha+\bar{\eta}\bar{v}}d\bar{t} = \frac{d\bar{v}}{\bar{v}} \]} 
\end{equation}

\end{document}
